\documentclass{article}

\usepackage{amssymb, amsmath}
\usepackage{alltt}
\usepackage{pslatex}
\usepackage{epigraph}
\usepackage{verbatim}
\usepackage{latexsym}
\usepackage{array}
\usepackage{comment}
\usepackage{makeidx}
\usepackage{listings}
\usepackage{indentfirst}
\usepackage{verbatim}
\usepackage{color}
\usepackage{url}
\usepackage{xspace}
\usepackage{hyperref}
\usepackage{stmaryrd}
\usepackage{amsmath, amsthm, amssymb}
\usepackage{graphicx}
\usepackage{euscript}
\usepackage{mathtools}
\usepackage{mathrsfs}
\usepackage{multirow,bigdelim}
\usepackage{subcaption}
\usepackage{placeins}

\makeatletter

\makeatother

\definecolor{shadecolor}{gray}{1.00}
\definecolor{darkgray}{gray}{0.30}

\def\transarrow{\xrightarrow}
\newcommand{\setarrow}[1]{\def\transarrow{#1}}

\def\padding{\phantom{X}}
\newcommand{\setpadding}[1]{\def\padding{#1}}

\def\subarrow{}
\newcommand{\setsubarrow}[1]{\def\subarrow{#1}}

\newcommand{\trule}[2]{\dfrac{#1}{#2}}
\newcommand{\crule}[3]{\frac{#1}{#2},\;{#3}}
\newcommand{\withenv}[2]{{#1}\vdash{#2}}
\newcommand{\trans}[3]{{#1}\transarrow{\padding{\textstyle #2}\padding}\subarrow{#3}}
\newcommand{\ctrans}[4]{{#1}\transarrow{\padding#2\padding}\subarrow{#3},\;{#4}}
\newcommand{\llang}[1]{\mbox{\lstinline[mathescape]|#1|}}
\newcommand{\pair}[2]{\inbr{{#1}\mid{#2}}}
\newcommand{\inbr}[1]{\left<{#1}\right>}
\newcommand{\highlight}[1]{\color{red}{#1}}
\newcommand{\ruleno}[1]{\eqno[\scriptsize\textsc{#1}]}
\newcommand{\rulename}[1]{\textsc{#1}}
\newcommand{\inmath}[1]{\mbox{$#1$}}
\newcommand{\lfp}[1]{fix_{#1}}
\newcommand{\gfp}[1]{Fix_{#1}}
\newcommand{\vsep}{\vspace{-2mm}}
\newcommand{\supp}[1]{\scriptsize{#1}}
\newcommand{\sembr}[1]{\llbracket{#1}\rrbracket}
\newcommand{\cd}[1]{\texttt{#1}}
\newcommand{\free}[1]{\boxed{#1}}
\newcommand{\binds}{\;\mapsto\;}
\newcommand{\dbi}[1]{\mbox{\bf{#1}}}
\newcommand{\sv}[1]{\mbox{\textbf{#1}}}
\newcommand{\bnd}[2]{{#1}\mkern-9mu\binds\mkern-9mu{#2}}
\newtheorem{lemma}{Lemma}
\newtheorem{theorem}{Theorem}
\newcommand{\meta}[1]{{\mathcal{#1}}}
\renewcommand{\emptyset}{\varnothing}
\newcommand{\dom}[1]{\mathtt{dom}\;{#1}}
\newcommand{\primi}[2]{\mathbf{#1}\;{#2}}

\definecolor{light-gray}{gray}{0.90}
\newcommand{\graybox}[1]{\colorbox{light-gray}{#1}}

\lstdefinelanguage{lama}{
keywords={read, write, skip,if,then,else,elif,fi,while,do,od,repeat,until,for,fun,local,public,return,import,length,
var,string,case,of,esac,when,boxed,unboxed,string,sexp,array,infix,infixl,infixr,at,before,after,true,false,eta,lazy,ignore, ref, elemRef},
sensitive=true,
basicstyle=\small,
%commentstyle=\scriptsize\rmfamily,
keywordstyle=\ttfamily\bfseries,
identifierstyle=\ttfamily,
basewidth={0.5em,0.5em},
columns=fixed,
fontadjust=true,
literate={->}{{$\to$}}3,
morecomment=[s][\ttfamily]{(*}{*)},
morecomment=[l][\ttfamily]{--}
}

\lstset{
mathescape=true,
basicstyle=\small,
identifierstyle=\ttfamily,
keywordstyle=\bfseries,
commentstyle=\scriptsize\rmfamily,
basewidth={0.5em,0.5em},
fontadjust=true,
escapechar=!,
language=lama
}

\sloppy

\newcommand{\lama}{$\lambda\kern -.1667em\lower -.5ex\hbox{$a$}\kern -.1000em\lower .2ex\hbox{$\mathcal M$}\kern -.1000em\lower -.5ex\hbox{$a$}$\xspace}

\theoremstyle{definition}

\author{Dmitry Boulytchev}

\begin{document}

\section{Pattern Matching}

A \emph{new} syntacic category: \emph{patterns} $\mathscr P$:

\[
\begin{array}{rcll}
  \mathscr P & = & \mathbb N & \mbox{--- number}\\
             &   & \_        & \mbox{--- wildcard}\\
             &   & \llang{[}\mathscr P^*\llang{]} & \mbox{--- array}\\
             &   & \mathscr C \mathscr P^* & \mbox{--- S-expression} \\
             &   & \mathscr X \llang{@} \mathscr P& \mbox{--- named pattern}
\end{array}
\]

Concrete syntax mainly repeats the abstract; in S-expression patterns extra round brackets are used to delimit the constructor's name from arguments (if any); arguments
of array/S-expression patterns are delimited by commas, and extra round brackets can be used to group subpatterns. Additionally, one derived form is used: an identifier
$x$ is treated as a pattern $x@\_$.

Pattern matching expression:

\[
\mathscr E += \llang{case}\;\; \mathscr E \;\llang{of}\;\; (\mathscr P \times \mathscr E)^+ \;\llang{esac}
\]

In a concrete syntax branches of case expression are delimited by ``\llang{|}'', and in each branch ``\llang{->}'' is used to delimit pattern from expression.

Well-formedness of case expressions is established in an obvious manner:

\ifdefined\Ref
  \renewcommand{\Ref}{\primi{Ref}{}}
\else
  \newcommand{\Ref}{\primi{Ref}{}}
\fi
\newcommand{\Val}{\primi{Val}{}}
\newcommand{\Void}{\primi{Void}{}}

\[
\renewcommand{\withenv}[2]{{#2}:{#1}}
\trule{\withenv{\Val}{e}\quad\withenv{a}{e_i}}
      {\withenv{a}{\llang{case $\;e\;$ of $\;p_1\;$ -> $\;e_1\;$ ... $\;p_k\;$ -> $\;e_k\;$ esac}}}
\]

\section{Operational Semantics}

There are two aspects that have to be covered in semantic description of pattern matching:

\begin{itemize}
\item the criterion for a \emph{scrutinee} to be matched by a pattern;
\item the descipline of binding support.
\end{itemize}

The latter aspect is covered by a desugaring. First, we define a mapping

\[
\beta : \mathscr P \to \mathscr E \to \mathscr X \to \mathscr E
\]

in a following manner:

\[
\begin{array}{rcl}
  \beta\;n\;e & = & \lambda \_ . \bot \\
  \beta\;\_\;e & = & \lambda \_ . \bot \\
  \beta\;([p_0\dots p_k])\;e & = & \\
  \beta\;(C\,p_0\dots p_k)\;e & = & (\beta\;p_0\;e[0])[\beta\;p_1\;e[1]]\dots [\beta\;p_k\;e[k]]\\
  \beta\;(x@p)\;e & = & (\beta\;p\;e) [x \gets e]\\
\end{array}
\]

This function determines a proper \emph{subvalue} of an expression bound by an identifier in a pattern. For example,
for a pattern $[x,\, C\, (\_,\, y)]$ and scrutenee $s$ the value of $\beta\, [x,\, C\, (\_,\, y)]\,s$ can be described by
the following table:

\[
\begin{array}{rcl}
  x & \to & s[0]\\
  y & \to & s[1][1]
\end{array}
\]

Then, given a pattern-matching expression \llang{case $\;\;e\;\;$ of ...} we, first, bind the expression $e$ to a \emph{fresh} variable,
say, $s$:

\begin{lstlisting}
  var $s$ = $e$;
  case $s$ of ...
\end{lstlisting}

Then, we transform each branch \llang{$p$ -> $e$} into the following:

\begin{lstlisting}
  $p$ -> var $b_1$ = $\beta\,p\,s\,b_1$,
             ...
           $b_k$ = $\beta\,p\,s\,b_k$;
           $e$
\end{lstlisting}

where $b_1,\dots,b_k$ are all bindings in $p$.

Now, for determining the descipline of matching we need an extra relation

\[
match \subseteq \mathscr P\times \mathscr V
\]

between patterns and values. We define it in a following way:

\[
\begin{array}{c}
match\,(\_,\,v)\\[2mm]
match\,(n,\,n)\\[4mm]
\trule{match\,(p_i,\,v_i)}
      {match\,([p_1,\dots,p_k],\,[v_1,\dots,v_k])}\\[5mm]
\trule{match\,(p_i,\,v_i)}
      {match\,(C\,p_1,\dots,p_k,\,C\,v_1,\dots,v_k)}\\[5mm]
\trule{match\,(p,\,v)}
      {match\,(x@p,\,v)}
\end{array}
\]

Finally, the operational semantics of pattern-matching expression can be given by the following
rules:

\[
\trule{\begin{array}{c}
        {\setarrow{\xRightarrow}\trans{c}{e}{\inbr{c^\prime,\,v}}}\\[2mm]
        {\setsubarrow{_{\mathscr P}}\withenv{v}{\trans{c^\prime}{(p_1,\,e_1)...(p_k,\,e_k)}{c^{\prime\prime}}}}
       \end{array}
      }
      {\setarrow{\xRightarrow}\trans{c}{\llang{case $\;e\;$ of $\;p_1$ -> $e_1\;$ ... $\;p_k$ -> $e_k\;$ esac}}{c^{\prime\prime}}}
\]
%       {\setarrow{\xRightarrow}\trans{\inbr{c,\,l}}{\llang{case $\;e\;$ of $\;p_1$ -> $e_1\;$ ... $\;p_k$ -> $e_k\;$ esac}}{c^{\prime\prime}}}
% \]

where an additional transition ``$\setsubarrow{_{\mathscr P}}\trans{}{}{}$'' is defined as follows:

\[
\begin{array}{c}
\trule{\begin{array}{c}
          match\,(p,\,v)\\
          {\setarrow{\xRightarrow}\trans{c}{e}{c^\prime}}
       \end{array}}
      {\setsubarrow{_{\mathscr P}}\withenv{v}{\trans{c}{(p,\,e)ps}{c^\prime}}}\\[10mm]
\trule{\begin{array}{c}
          \neg match\,(p,\,v)\\
          \withenv{v}{\setsubarrow{_{\mathscr P}}\trans{c}{ps}{c^\prime}}
       \end{array}}
      {\setsubarrow{_{\mathscr P}}\withenv{v}{\trans{c}{(p,\,e)ps}{c^\prime}}}
\end{array}
\]

\end{document}
